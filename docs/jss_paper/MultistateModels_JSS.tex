% Options for packages loaded elsewhere
% Options for packages loaded elsewhere
\PassOptionsToPackage{unicode}{hyperref}
\PassOptionsToPackage{hyphens}{url}
\PassOptionsToPackage{dvipsnames,svgnames,x11names}{xcolor}
%
\documentclass[
  article]{jss}
\usepackage{xcolor}
\usepackage{amsmath,amssymb}
\setcounter{secnumdepth}{-\maxdimen} % remove section numbering
\usepackage{iftex}
\ifPDFTeX
  \usepackage[T1]{fontenc}
  \usepackage[utf8]{inputenc}
  \usepackage{textcomp} % provide euro and other symbols
\else % if luatex or xetex
  \usepackage{unicode-math} % this also loads fontspec
  \defaultfontfeatures{Scale=MatchLowercase}
  \defaultfontfeatures[\rmfamily]{Ligatures=TeX,Scale=1}
\fi
\usepackage{lmodern}
\ifPDFTeX\else
  % xetex/luatex font selection
\fi
% Use upquote if available, for straight quotes in verbatim environments
\IfFileExists{upquote.sty}{\usepackage{upquote}}{}
\IfFileExists{microtype.sty}{% use microtype if available
  \usepackage[]{microtype}
  \UseMicrotypeSet[protrusion]{basicmath} % disable protrusion for tt fonts
}{}
\makeatletter
\@ifundefined{KOMAClassName}{% if non-KOMA class
  \IfFileExists{parskip.sty}{%
    \usepackage{parskip}
  }{% else
    \setlength{\parindent}{0pt}
    \setlength{\parskip}{6pt plus 2pt minus 1pt}}
}{% if KOMA class
  \KOMAoptions{parskip=half}}
\makeatother
% Make \paragraph and \subparagraph free-standing
\makeatletter
\ifx\paragraph\undefined\else
  \let\oldparagraph\paragraph
  \renewcommand{\paragraph}{
    \@ifstar
      \xxxParagraphStar
      \xxxParagraphNoStar
  }
  \newcommand{\xxxParagraphStar}[1]{\oldparagraph*{#1}\mbox{}}
  \newcommand{\xxxParagraphNoStar}[1]{\oldparagraph{#1}\mbox{}}
\fi
\ifx\subparagraph\undefined\else
  \let\oldsubparagraph\subparagraph
  \renewcommand{\subparagraph}{
    \@ifstar
      \xxxSubParagraphStar
      \xxxSubParagraphNoStar
  }
  \newcommand{\xxxSubParagraphStar}[1]{\oldsubparagraph*{#1}\mbox{}}
  \newcommand{\xxxSubParagraphNoStar}[1]{\oldsubparagraph{#1}\mbox{}}
\fi
\makeatother

\usepackage{color}
\usepackage{fancyvrb}
\newcommand{\VerbBar}{|}
\newcommand{\VERB}{\Verb[commandchars=\\\{\}]}
\DefineVerbatimEnvironment{Highlighting}{Verbatim}{commandchars=\\\{\}}
% Add ',fontsize=\small' for more characters per line
\usepackage{framed}
\definecolor{shadecolor}{RGB}{241,243,245}
\newenvironment{Shaded}{\begin{snugshade}}{\end{snugshade}}
\newcommand{\AlertTok}[1]{\textcolor[rgb]{0.68,0.00,0.00}{#1}}
\newcommand{\AnnotationTok}[1]{\textcolor[rgb]{0.37,0.37,0.37}{#1}}
\newcommand{\AttributeTok}[1]{\textcolor[rgb]{0.40,0.45,0.13}{#1}}
\newcommand{\BaseNTok}[1]{\textcolor[rgb]{0.68,0.00,0.00}{#1}}
\newcommand{\BuiltInTok}[1]{\textcolor[rgb]{0.00,0.23,0.31}{#1}}
\newcommand{\CharTok}[1]{\textcolor[rgb]{0.13,0.47,0.30}{#1}}
\newcommand{\CommentTok}[1]{\textcolor[rgb]{0.37,0.37,0.37}{#1}}
\newcommand{\CommentVarTok}[1]{\textcolor[rgb]{0.37,0.37,0.37}{\textit{#1}}}
\newcommand{\ConstantTok}[1]{\textcolor[rgb]{0.56,0.35,0.01}{#1}}
\newcommand{\ControlFlowTok}[1]{\textcolor[rgb]{0.00,0.23,0.31}{\textbf{#1}}}
\newcommand{\DataTypeTok}[1]{\textcolor[rgb]{0.68,0.00,0.00}{#1}}
\newcommand{\DecValTok}[1]{\textcolor[rgb]{0.68,0.00,0.00}{#1}}
\newcommand{\DocumentationTok}[1]{\textcolor[rgb]{0.37,0.37,0.37}{\textit{#1}}}
\newcommand{\ErrorTok}[1]{\textcolor[rgb]{0.68,0.00,0.00}{#1}}
\newcommand{\ExtensionTok}[1]{\textcolor[rgb]{0.00,0.23,0.31}{#1}}
\newcommand{\FloatTok}[1]{\textcolor[rgb]{0.68,0.00,0.00}{#1}}
\newcommand{\FunctionTok}[1]{\textcolor[rgb]{0.28,0.35,0.67}{#1}}
\newcommand{\ImportTok}[1]{\textcolor[rgb]{0.00,0.46,0.62}{#1}}
\newcommand{\InformationTok}[1]{\textcolor[rgb]{0.37,0.37,0.37}{#1}}
\newcommand{\KeywordTok}[1]{\textcolor[rgb]{0.00,0.23,0.31}{\textbf{#1}}}
\newcommand{\NormalTok}[1]{\textcolor[rgb]{0.00,0.23,0.31}{#1}}
\newcommand{\OperatorTok}[1]{\textcolor[rgb]{0.37,0.37,0.37}{#1}}
\newcommand{\OtherTok}[1]{\textcolor[rgb]{0.00,0.23,0.31}{#1}}
\newcommand{\PreprocessorTok}[1]{\textcolor[rgb]{0.68,0.00,0.00}{#1}}
\newcommand{\RegionMarkerTok}[1]{\textcolor[rgb]{0.00,0.23,0.31}{#1}}
\newcommand{\SpecialCharTok}[1]{\textcolor[rgb]{0.37,0.37,0.37}{#1}}
\newcommand{\SpecialStringTok}[1]{\textcolor[rgb]{0.13,0.47,0.30}{#1}}
\newcommand{\StringTok}[1]{\textcolor[rgb]{0.13,0.47,0.30}{#1}}
\newcommand{\VariableTok}[1]{\textcolor[rgb]{0.07,0.07,0.07}{#1}}
\newcommand{\VerbatimStringTok}[1]{\textcolor[rgb]{0.13,0.47,0.30}{#1}}
\newcommand{\WarningTok}[1]{\textcolor[rgb]{0.37,0.37,0.37}{\textit{#1}}}

\usepackage{longtable,booktabs,array}
\usepackage{calc} % for calculating minipage widths
% Correct order of tables after \paragraph or \subparagraph
\usepackage{etoolbox}
\makeatletter
\patchcmd\longtable{\par}{\if@noskipsec\mbox{}\fi\par}{}{}
\makeatother
% Allow footnotes in longtable head/foot
\IfFileExists{footnotehyper.sty}{\usepackage{footnotehyper}}{\usepackage{footnote}}
\makesavenoteenv{longtable}
\usepackage{graphicx}
\makeatletter
\newsavebox\pandoc@box
\newcommand*\pandocbounded[1]{% scales image to fit in text height/width
  \sbox\pandoc@box{#1}%
  \Gscale@div\@tempa{\textheight}{\dimexpr\ht\pandoc@box+\dp\pandoc@box\relax}%
  \Gscale@div\@tempb{\linewidth}{\wd\pandoc@box}%
  \ifdim\@tempb\p@<\@tempa\p@\let\@tempa\@tempb\fi% select the smaller of both
  \ifdim\@tempa\p@<\p@\scalebox{\@tempa}{\usebox\pandoc@box}%
  \else\usebox{\pandoc@box}%
  \fi%
}
% Set default figure placement to htbp
\def\fps@figure{htbp}
\makeatother





\setlength{\emergencystretch}{3em} % prevent overfull lines

\providecommand{\tightlist}{%
  \setlength{\itemsep}{0pt}\setlength{\parskip}{0pt}}



 
\usepackage[]{natbib}
\bibliographystyle{plainnat}


\usepackage{amsmath}
\usepackage{amssymb}
\usepackage{algorithm}
\usepackage{algpseudocode}
\usepackage{booktabs}
\usepackage{longtable}
\makeatletter
\@ifpackageloaded{caption}{}{\usepackage{caption}}
\AtBeginDocument{%
\ifdefined\contentsname
  \renewcommand*\contentsname{Table of contents}
\else
  \newcommand\contentsname{Table of contents}
\fi
\ifdefined\listfigurename
  \renewcommand*\listfigurename{List of Figures}
\else
  \newcommand\listfigurename{List of Figures}
\fi
\ifdefined\listtablename
  \renewcommand*\listtablename{List of Tables}
\else
  \newcommand\listtablename{List of Tables}
\fi
\ifdefined\figurename
  \renewcommand*\figurename{Figure}
\else
  \newcommand\figurename{Figure}
\fi
\ifdefined\tablename
  \renewcommand*\tablename{Table}
\else
  \newcommand\tablename{Table}
\fi
}
\@ifpackageloaded{float}{}{\usepackage{float}}
\floatstyle{ruled}
\@ifundefined{c@chapter}{\newfloat{codelisting}{h}{lop}}{\newfloat{codelisting}{h}{lop}[chapter]}
\floatname{codelisting}{Listing}
\newcommand*\listoflistings{\listof{codelisting}{List of Listings}}
\makeatother
\makeatletter
\makeatother
\makeatletter
\@ifpackageloaded{caption}{}{\usepackage{caption}}
\@ifpackageloaded{subcaption}{}{\usepackage{subcaption}}
\makeatother
\usepackage{bookmark}
\IfFileExists{xurl.sty}{\usepackage{xurl}}{} % add URL line breaks if available
\urlstyle{same}
\hypersetup{
  pdftitle={MultistateModels.jl: Fitting Semi-Markov Multistate Models to Panel Data in Julia},
  pdfauthor={Jon Fintz; Raphael Morsomme; Jason Liang; Allyson Mateja},
  pdfkeywords={multistate models, semi-Markov, panel data, interval
censoring, Monte Carlo EM, Julia},
  colorlinks=true,
  linkcolor={blue},
  filecolor={Maroon},
  citecolor={Blue},
  urlcolor={Blue},
  pdfcreator={LaTeX via pandoc}}


\title{MultistateModels.jl: Fitting Semi-Markov Multistate Models to
Panel Data in Julia}
\author{Jon Fintz \and Raphael Morsomme \and Jason Liang \and Allyson Mateja}
\date{}
\begin{document}
\maketitle
\begin{abstract}
We present \textbf{MultistateModels.jl}, a Julia package for fitting
continuous-time multistate models to exact and interval-censored (panel)
data. The package implements and extends the methods of
\citet{morsomme2025semimarkov}, providing a unified interface for Markov
and semi-Markov multistate models with flexible hazard specifications
including exponential, Weibull, Gompertz, B-spline, and phase-type
(Coxian) distributions. A key innovation is the Monte Carlo
Expectation-Maximization (MCEM) algorithm that enables fitting
semi-Markov models to panel data---a capability not available in
existing software. The package supports proportional hazards models with
time-varying covariates, multiple variance estimation methods
(model-based, infinitesimal jackknife, and jackknife), and efficient
importance sampling with Pareto-smoothed diagnostics. We demonstrate the
package through clinical examples and provide systematic comparisons
with existing R packages including \textbf{msm}, \textbf{mstate},
\textbf{SemiMarkov}, and \textbf{flexsurv}.
\end{abstract}


\begin{Shaded}
\begin{Highlighting}[]
\NormalTok{\#| echo: false}
\NormalTok{\#| output: false}
\NormalTok{\#| eval: false}
\NormalTok{using MultistateModels}
\NormalTok{using DataFrames}
\NormalTok{using Random}
\NormalTok{using Statistics}
\end{Highlighting}
\end{Shaded}

\section{Introduction}\label{sec-introduction}

Multistate models provide a powerful framework for analyzing event
history data where individuals transition between discrete states over
time. These models are ubiquitous in biomedical research, including
disease progression studies, treatment response evaluation, and health
services research. The defining characteristic of multistate models is
their ability to capture the temporal dynamics of state occupancy,
competing risks, and the effects of covariates on transition
intensities.

\subsection{Motivation}\label{motivation}

In many practical applications, the exact times of state transitions are
not observed. Instead, an individual's state is recorded only at
discrete observation times---a data structure known as \emph{panel data}
or \emph{interval-censored observations}. For example, in clinical
trials, a patient's disease status may be assessed at scheduled clinic
visits (e.g., every 3 months), but the precise timing of disease
progression between visits is unknown.

The standard approach to analyzing panel data is to assume the
underlying process is \emph{Markov}, meaning that future transitions
depend only on the current state, not on how long the process has been
in that state. Under this assumption, the likelihood can be computed
efficiently using matrix exponentials of the transition intensity matrix
\citep{jackson2011msm}. However, the Markov assumption implies
exponentially distributed sojourn times, which may be inappropriate when
the hazard of leaving a state increases or decreases with time spent in
that state.

\emph{Semi-Markov} models relax this assumption by allowing transition
intensities to depend on the \emph{sojourn time} (time since entering
the current state). This enables more flexible modeling of holding time
distributions, but creates a fundamental computational challenge: the
likelihood for panel data requires marginalizing over all possible
latent paths between observation times, an intractable high-dimensional
integral.

\begin{figure}

\centering{

\pandocbounded{\includegraphics[keepaspectratio]{figures/panel_data.pdf}}

}

\caption{\label{fig-panel-data}Exact path vs.~panel observations. The
true process (black line) transitions continuously between states, but
is only observed at discrete time points (red dots). The likelihood
requires integrating over all possible paths consistent with these
observations.}

\end{figure}%

\subsection{Existing Software}\label{existing-software}

Several software packages implement multistate models, but none provides
the combination of semi-Markov dynamics and panel data support offered
by \textbf{MultistateModels.jl}. Table~\ref{tbl-software-comparison}
summarizes the capabilities of existing packages.

\begin{longtable}[]{@{}
  >{\raggedright\arraybackslash}p{(\linewidth - 10\tabcolsep) * \real{0.1071}}
  >{\raggedright\arraybackslash}p{(\linewidth - 10\tabcolsep) * \real{0.1190}}
  >{\raggedright\arraybackslash}p{(\linewidth - 10\tabcolsep) * \real{0.1786}}
  >{\raggedright\arraybackslash}p{(\linewidth - 10\tabcolsep) * \real{0.1190}}
  >{\raggedright\arraybackslash}p{(\linewidth - 10\tabcolsep) * \real{0.1548}}
  >{\raggedright\arraybackslash}p{(\linewidth - 10\tabcolsep) * \real{0.3214}}@{}}
\caption{Comparison of multistate modeling software. IC =
interval-censored.}\label{tbl-software-comparison}\tabularnewline
\toprule\noalign{}
\begin{minipage}[b]{\linewidth}\raggedright
Package
\end{minipage} & \begin{minipage}[b]{\linewidth}\raggedright
Language
\end{minipage} & \begin{minipage}[b]{\linewidth}\raggedright
Primary Focus
\end{minipage} & \begin{minipage}[b]{\linewidth}\raggedright
Panel/IC
\end{minipage} & \begin{minipage}[b]{\linewidth}\raggedright
Semi-Markov
\end{minipage} & \begin{minipage}[b]{\linewidth}\raggedright
Parametric/Semi-parametric
\end{minipage} \\
\midrule\noalign{}
\endfirsthead
\toprule\noalign{}
\begin{minipage}[b]{\linewidth}\raggedright
Package
\end{minipage} & \begin{minipage}[b]{\linewidth}\raggedright
Language
\end{minipage} & \begin{minipage}[b]{\linewidth}\raggedright
Primary Focus
\end{minipage} & \begin{minipage}[b]{\linewidth}\raggedright
Panel/IC
\end{minipage} & \begin{minipage}[b]{\linewidth}\raggedright
Semi-Markov
\end{minipage} & \begin{minipage}[b]{\linewidth}\raggedright
Parametric/Semi-parametric
\end{minipage} \\
\midrule\noalign{}
\endhead
\bottomrule\noalign{}
\endlastfoot
\textbf{survival} \citep{therneau2000survival} & R & Cox regression,
Kaplan-Meier & ✗ & ✗ & Semi-parametric (Cox) \\
\textbf{msm} \citep{jackson2011msm} & R & Markov, HMM panel data & ✓ & ✗
& Parametric \\
\textbf{mstate} \citep{putter2007tutorial} & R & Non-parametric,
Aalen-Johansen & ✗ & ✗ & Non-parametric \\
\textbf{etm} \citep{allignol2011etm} & R & Non-parametric empirical
transitions & ✗ & ✗ & Non-parametric \\
\textbf{SemiMarkov} \citep{krol2015semimarkov} & R & Semi-Markov
parametric & ✗ & ✓ & Parametric \\
\textbf{flexsurv} \citep{jackson2016flexsurv} & R & Parametric, splines
& ✗ & ✓ & Parametric + semi-parametric \\
\textbf{icmstate} \citep{gomon2024icmstate} & R & Non-parametric panel
data & ✓ & ✗ & Non-parametric \\
\textbf{MultistateModels.jl} & Julia &
\textbf{Parametric/semi-parametric semi-Markov + panel} & \textbf{✓} &
\textbf{✓} & \textbf{Parametric + semi-parametric} \\
\end{longtable}

The \textbf{msm} package \citep{jackson2011msm} is widely used for
fitting continuous-time Markov models to panel data, with support for
hidden Markov models and time-varying covariates. However, it is
restricted to the Markov assumption and cannot accommodate semi-Markov
dynamics.

The \textbf{SemiMarkov} package \citep{krol2015semimarkov} fits
semi-Markov models with Weibull or exponentiated Weibull sojourn
distributions, but it \textbf{requires exact observation of transition
times}---it cannot handle panel data where transition times are
interval-censored.

The foundational \textbf{survival} package \citep{therneau2000survival}
provides Cox regression for single-endpoint and competing risks models,
along with Kaplan-Meier and Nelson-Aalen estimators. While not
specifically designed for multistate models, it is the standard tool for
semi-parametric survival analysis in R.

The \textbf{mstate} package \citep{putter2007tutorial} extends survival
analysis to multistate models, providing non-parametric (Aalen-Johansen)
and Cox regression-based methods for estimating transition probabilities
and cumulative hazards. The \textbf{etm} package \citep{allignol2011etm}
provides efficient computation of empirical transition matrices. Both
require exact transition times and do not support fully parametric
inference. \textbf{flexsurv} \citep{jackson2016flexsurv} offers flexible
parametric and spline-based hazards for semi-Markov multistate models,
but requires exactly observed event times.

The recent \textbf{icmstate} package \citep{gomon2024icmstate} addresses
interval-censored multistate models using non-parametric methods, but is
limited to Markov models and does not support parametric or
semi-parametric inference.

\subsection{Related Work}\label{related-work}

Our methodology builds on the hidden semi-Markov model literature.
\citet{alaa2018hasmm} developed the Hidden Absorbing Semi-Markov Model
(HASMM) for electronic health record data, using a forward-filtering
backward-sampling (FFBS) Monte Carlo EM algorithm similar to our
approach. Their work focused on hidden states and informative censoring
in clinical risk prognosis, while our package targets the classical
multistate model setting with observed state occupancy at panel times.

The theoretical foundations of our MCEM algorithm are established in
\citet{morsomme2025semimarkov}, which provides convergence guarantees
and develops the importance sampling strategies implemented in this
package.

\subsection{Package Contributions}\label{package-contributions}

\textbf{MultistateModels.jl} makes the following contributions:

\begin{enumerate}
\def\labelenumi{\arabic{enumi}.}
\item
  \textbf{Unified interface}: A single \texttt{fit()} function
  automatically selects the appropriate inference method (direct MLE,
  matrix exponential, or MCEM) based on the model and data
  characteristics.
\item
  \textbf{Semi-Markov models for panel data}: The first software
  implementation enabling parametric and semi-parametric (B-spline)
  semi-Markov models to be fit to interval-censored observations via
  MCEM.
\item
  \textbf{Flexible hazard specifications}: Support for exponential,
  Weibull, Gompertz, B-spline, and phase-type (Coxian) hazard functions,
  all with optional covariate effects.
\item
  \textbf{Multiple variance estimators}: Model-based (observed Fisher
  information), infinitesimal jackknife (sandwich), and jackknife
  variance estimation.
\item
  \textbf{Efficient implementation}: Julia's multiple dispatch and
  just-in-time compilation provide performance comparable to compiled
  languages while maintaining high-level expressiveness.
\end{enumerate}

\subsection{Paper Organization}\label{paper-organization}

The remainder of this paper is organized as follows.
Section~\ref{sec-framework} presents the mathematical framework for
multistate models, including Markov and semi-Markov formulations.
Section~\ref{sec-construction} describes model construction in
\textbf{MultistateModels.jl}. Section~\ref{sec-simulation} covers the
simulation functionality. Section~\ref{sec-inference} details the
inference algorithms, including the MCEM procedure for semi-Markov
models with panel data. Section~\ref{sec-model-selection} discusses
model selection approaches. Section~\ref{sec-computational} provides
computational details. Section~\ref{sec-outputs} describes model outputs
and diagnostics. Section~\ref{sec-examples} presents worked examples.
Section~\ref{sec-comparison} provides detailed comparisons with existing
software. Section~\ref{sec-conclusion} concludes.

\section{Mathematical Framework}\label{sec-framework}

\subsection{Multistate Processes}\label{multistate-processes}

A multistate process \(\{X(t) : t \geq 0\}\) is a continuous-time
stochastic process taking values in a finite state space
\(\mathcal{S} = \{1, 2, \ldots, K\}\). Some states may be
\emph{absorbing} (e.g., death), meaning no transitions out of that state
are possible. The process is characterized by its transition intensities
or \emph{hazard functions}.

For a subject currently in state \(j\) at time \(t\), the
\emph{cause-specific hazard} for transition to state \(k \neq j\) is:
\begin{equation}\phantomsection\label{eq-hazard-def}{
\lambda_{jk}(t \mid \mathcal{H}_t) = \lim_{h \downarrow 0} 
\frac{P(X(t+h) = k \mid X(t) = j, \mathcal{H}_t)}{h}
}\end{equation} where \(\mathcal{H}_t\) denotes the history of the
process up to time \(t\).

The \emph{total hazard} of leaving state \(j\) is: \[
\lambda_j(t \mid \mathcal{H}_t) = \sum_{k \neq j} \lambda_{jk}(t \mid \mathcal{H}_t)
\]

\subsection{Markov vs.~Semi-Markov
Models}\label{markov-vs.-semi-markov-models}

In a \textbf{Markov} multistate model, the hazards depend only on the
current state and calendar time \(t\): \[
\lambda_{jk}(t \mid \mathcal{H}_t) = \lambda_{jk}(t)
\] This implies that the sojourn time in any state follows an
exponential distribution (possibly with time-varying rate), and the
process is memoryless given the current state.

In a \textbf{semi-Markov} model, the hazards depend on the \emph{sojourn
time} \(u\)---the time elapsed since entering the current state:
\begin{equation}\phantomsection\label{eq-semimarkov-hazard}{
\lambda_{jk}(u) = \lim_{h \downarrow 0} 
\frac{P(\text{transition to } k \text{ in } (u, u+h] \mid \text{in state } j \text{ for time } u)}{h}
}\end{equation} The clock ``resets'' at each transition, allowing
flexible modeling of state-specific holding time distributions.

\subsection{Parametric Hazard
Families}\label{parametric-hazard-families}

\textbf{MultistateModels.jl} supports several parametric families for
the baseline hazard functions:

\textbf{Exponential} (\texttt{:exp}): Constant hazard, yielding
exponentially distributed sojourn times: \[
\lambda(t) = \exp(\beta_0)
\]

\textbf{Weibull} (\texttt{:wei}): Shape-scale parameterization allowing
increasing (\(\kappa > 1\)) or decreasing (\(\kappa < 1\)) hazards: \[
\lambda(t) = \frac{\kappa}{\sigma}\left(\frac{t}{\sigma}\right)^{\kappa-1}
\] with log-parameterization \((\log \kappa, \log \sigma)\) for
unconstrained optimization.

\textbf{Gompertz} (\texttt{:gom}): Exponentially increasing or
decreasing hazard: \[
\lambda(t) = \exp(\beta_0 + \beta_1 t)
\]

\textbf{B-Spline} (\texttt{:sp}): Flexible semi-parametric hazard using
spline basis functions: \[
\lambda(t) = \exp\left(\sum_{j=1}^{J} \beta_j B_j(t)\right)
\] where \(B_j(t)\) are B-spline basis functions. Monotonicity
constraints can be imposed using I-spline representations.

\textbf{Phase-Type} (\texttt{:pt}): Coxian phase-type distributions that
can approximate any sojourn distribution through a latent Markov chain
on ``phases.'' See Section~\ref{sec-phasetype} for details.

\begin{figure}

\centering{

\pandocbounded{\includegraphics[keepaspectratio]{figures/hazards.pdf}}

}

\caption{\label{fig-hazards}Flexible hazard specifications supported by
MultistateModels.jl. The package supports standard parametric families
like Weibull (monotonic) as well as flexible B-splines (non-monotonic)
and phase-type distributions.}

\end{figure}%

\subsection{Covariate Effects}\label{covariate-effects}

Covariates enter the model through the proportional hazards formulation:
\begin{equation}\phantomsection\label{eq-prop-hazards}{
\lambda_{jk}(t \mid \mathbf{x}) = \lambda_{jk,0}(t) \exp(\boldsymbol{\beta}_{jk}^\top \mathbf{x})
}\end{equation} where \(\lambda_{jk,0}(t)\) is the baseline hazard and
\(\mathbf{x}\) is a vector of covariates. Time-varying covariates are
supported through piecewise-constant approximations.

\subsection{Likelihood Formulations}\label{likelihood-formulations}

\subsubsection{Exact Data}\label{exact-data}

When transition times are exactly observed, the likelihood for subject
\(i\) with path
\((j_1, t_1) \to (j_2, t_2) \to \cdots \to (j_{M_i}, t_{M_i})\) is:
\begin{equation}\phantomsection\label{eq-exact-lik}{
\mathcal{L}_i(\theta) = \prod_{m=1}^{M_i-1} \lambda_{j_m j_{m+1}}(u_m) 
\exp\left(-\int_0^{u_m} \lambda_{j_m}(s) ds\right)
}\end{equation} where \(u_m = t_{m+1} - t_m\) is the sojourn time in
state \(j_m\).

\subsubsection{Panel Data (Markov)}\label{panel-data-markov}

For panel data where state is observed at times
\(0 = t_0 < t_1 < \cdots < t_M\), but transitions between observations
are unobserved, the likelihood under the Markov assumption is:
\begin{equation}\phantomsection\label{eq-panel-markov-lik}{
\mathcal{L}_i(\theta) = \prod_{m=1}^{M_i} P_{j_{m-1} j_m}(t_{m-1}, t_m; \theta)
}\end{equation} where
\(P_{jk}(s, t; \theta) = P(X(t) = k \mid X(s) = j)\) is the transition
probability, computed via matrix exponential: \[
\mathbf{P}(s, t) = \exp\left(\int_s^t \mathbf{Q}(u) du\right)
\] with \(\mathbf{Q}(t)\) the intensity matrix having off-diagonal
elements \(Q_{jk}(t) = \lambda_{jk}(t)\) and diagonal elements
\(Q_{jj}(t) = -\lambda_j(t)\).

\subsubsection{Panel Data (Semi-Markov)}\label{panel-data-semi-markov}

For semi-Markov models with panel data, the likelihood requires
integrating over all possible latent paths:
\begin{equation}\phantomsection\label{eq-panel-semimarkov-lik}{
\mathcal{L}_i(\theta) = \sum_{\text{paths } \pi} P(\pi \mid \theta) 
\cdot \mathbf{1}[\pi \text{ compatible with observations}]
}\end{equation} This integral is generally intractable, motivating the
MCEM approach described in Section~\ref{sec-mcem}.

\subsection{Phase-Type Distributions}\label{sec-phasetype}

Phase-type distributions provide a flexible class of sojourn
distributions through latent Markov structure. A \emph{Coxian}
phase-type distribution represents the sojourn time as absorption time
in a Markov chain on phases \(\{1, 2, \ldots, n\}\) with:

\begin{itemize}
\tightlist
\item
  Initial state: always phase 1
\item
  Transitions: phase \(i\) can transition to phase \(i+1\) or absorb
  (exit)
\item
  Absorption represents completion of the sojourn
\end{itemize}

The hazard function for a Coxian distribution can approximate any
continuous hazard arbitrarily well as the number of phases increases
\citep{asmussen1996fitting}. This provides a non-parametric approach to
sojourn time modeling while maintaining the Markov property at the phase
level---enabling efficient likelihood computation even for panel data.

\section{Model Construction}\label{sec-construction}

\subsection{The Hazard Constructor}\label{the-hazard-constructor}

Hazard specifications are created using the \texttt{Hazard()} function:

\begin{Shaded}
\begin{Highlighting}[]
\CommentTok{\# Exponential hazard for transition 1→2}
\NormalTok{h12 }\OperatorTok{=} \FunctionTok{Hazard}\NormalTok{(}\OperatorTok{:}\NormalTok{exp, }\FloatTok{1}\NormalTok{, }\FloatTok{2}\NormalTok{)}

\CommentTok{\# Weibull hazard for transition 1→3 with covariates}
\NormalTok{h13 }\OperatorTok{=} \FunctionTok{Hazard}\NormalTok{(}\PreprocessorTok{@formula}\NormalTok{(}\FloatTok{0} \OperatorTok{\textasciitilde{}}\NormalTok{ age }\OperatorTok{+}\NormalTok{ treatment), }\OperatorTok{:}\NormalTok{wei, }\FloatTok{1}\NormalTok{, }\FloatTok{3}\NormalTok{)}

\CommentTok{\# Spline hazard with monotonicity constraint}
\NormalTok{h23 }\OperatorTok{=} \FunctionTok{Hazard}\NormalTok{(}\OperatorTok{:}\NormalTok{sp, }\FloatTok{2}\NormalTok{, }\FloatTok{3}\NormalTok{; degree}\OperatorTok{=}\FloatTok{3}\NormalTok{, knots}\OperatorTok{=}\NormalTok{[}\FloatTok{1.0}\NormalTok{, }\FloatTok{2.0}\NormalTok{, }\FloatTok{5.0}\NormalTok{], monotone}\OperatorTok{=}\FloatTok{1}\NormalTok{)}

\CommentTok{\# Phase{-}type hazard (n\_phases specified at model level)}
\NormalTok{h12\_pt }\OperatorTok{=} \FunctionTok{Hazard}\NormalTok{(}\OperatorTok{:}\NormalTok{pt, }\FloatTok{1}\NormalTok{, }\FloatTok{2}\NormalTok{)}
\end{Highlighting}
\end{Shaded}

The first argument is an optional \texttt{@formula} specifying
covariates that act multiplicatively on the baseline hazard. When
omitted, an intercept-only model is assumed. The \texttt{family}
argument specifies the hazard family (\texttt{:exp}, \texttt{:wei},
\texttt{:gom}, \texttt{:sp}, or \texttt{:pt}), and
\texttt{statefrom}/\texttt{stateto} define the transition.

\subsubsection{Spline Hazards}\label{spline-hazards}

For spline hazards, additional keyword arguments control the basis:

\begin{itemize}
\tightlist
\item
  \texttt{degree}: Polynomial degree (default 3 for cubic splines)
\item
  \texttt{knots}: Interior knot locations
\item
  \texttt{boundaryknots}: Boundary knot locations (default to data
  range)
\item
  \texttt{natural\_spline}: Use natural spline boundary conditions
  (default \texttt{true})
\item
  \texttt{monotone}: Monotonicity constraint (-1 decreasing, 0 none, 1
  increasing)
\end{itemize}

\subsubsection{Phase-Type Hazards}\label{phase-type-hazards}

Phase-type hazards are specified with \texttt{family\ =\ :pt}. The
number of phases is typically set at the model level:

\begin{Shaded}
\begin{Highlighting}[]
\NormalTok{model }\OperatorTok{=} \FunctionTok{multistatemodel}\NormalTok{(h12\_pt, h23; data}\OperatorTok{=}\NormalTok{df, }
\NormalTok{                        n\_phases}\OperatorTok{=}\FunctionTok{Dict}\NormalTok{(}\FloatTok{1} \OperatorTok{=\textgreater{}} \FloatTok{3}\NormalTok{, }\FloatTok{2} \OperatorTok{=\textgreater{}} \FloatTok{2}\NormalTok{))}
\end{Highlighting}
\end{Shaded}

\subsection{The multistatemodel
Function}\label{the-multistatemodel-function}

Models are assembled using \texttt{multistatemodel()}:

\begin{Shaded}
\begin{Highlighting}[]
\NormalTok{model }\OperatorTok{=} \FunctionTok{multistatemodel}\NormalTok{(h12, h13, h23; }
\NormalTok{                        data }\OperatorTok{=}\NormalTok{ panel\_data,}
\NormalTok{                        surrogate }\OperatorTok{=} \OperatorTok{:}\NormalTok{markov)}
\end{Highlighting}
\end{Shaded}

Key arguments:

\begin{itemize}
\tightlist
\item
  \textbf{Hazard specifications}: Variable number of \texttt{Hazard}
  objects
\item
  \texttt{data}: A \texttt{DataFrame} with required columns
\item
  \texttt{surrogate}: Surrogate model for MCEM (\texttt{:markov} or
  \texttt{:phasetype})
\item
  \texttt{n\_phases}: Phase counts for phase-type hazards
\end{itemize}

\subsection{Data Format}\label{data-format}

The data \texttt{DataFrame} must contain:

\begin{longtable}[]{@{}ll@{}}
\caption{Required data columns.}\label{tbl-data-format}\tabularnewline
\toprule\noalign{}
Column & Description \\
\midrule\noalign{}
\endfirsthead
\toprule\noalign{}
Column & Description \\
\midrule\noalign{}
\endhead
\bottomrule\noalign{}
\endlastfoot
\texttt{id} & Subject identifier \\
\texttt{tstart} & Interval start time \\
\texttt{tstop} & Interval end time \\
\texttt{statefrom} & State at \texttt{tstart} \\
\texttt{stateto} & State at \texttt{tstop} (0 for censored) \\
\texttt{obstype} & Observation type: 1=exact, 2=panel, 3=censored \\
\end{longtable}

Additional columns for covariates are included as needed.

\textbf{Observation types:}

\begin{itemize}
\tightlist
\item
  \texttt{obstype\ =\ 1}: Exact transition observed at \texttt{tstop}
\item
  \texttt{obstype\ =\ 2}: Panel observation---state at \texttt{tstop}
  known, but intervening transitions unobserved
\item
  \texttt{obstype\ =\ 3}: Right-censored---subject in state
  \texttt{statefrom} at \texttt{tstart}, censored thereafter
\end{itemize}

\section{Simulation}\label{sec-simulation}

\subsection{Path Simulation}\label{path-simulation}

The \texttt{simulate()} function generates sample paths from a fitted or
parameterized model:

\begin{Shaded}
\begin{Highlighting}[]
\CommentTok{\# Set parameters}
\FunctionTok{set\_parameters!}\NormalTok{(model, (}
\NormalTok{    h12 }\OperatorTok{=}\NormalTok{ [}\FunctionTok{log}\NormalTok{(}\FloatTok{0.1}\NormalTok{)],        }\CommentTok{\# Exponential rate}
\NormalTok{    h13 }\OperatorTok{=}\NormalTok{ [}\FunctionTok{log}\NormalTok{(}\FloatTok{1.2}\NormalTok{), }\FunctionTok{log}\NormalTok{(}\FloatTok{0.5}\NormalTok{), }\FloatTok{0.3}\NormalTok{, }\OperatorTok{{-}}\FloatTok{0.2}\NormalTok{],  }\CommentTok{\# Weibull + covariates}
\NormalTok{    h23 }\OperatorTok{=}\NormalTok{ [}\FunctionTok{log}\NormalTok{(}\FloatTok{0.2}\NormalTok{)]}
\NormalTok{))}

\CommentTok{\# Simulate paths}
\NormalTok{sim\_result }\OperatorTok{=} \FunctionTok{simulate}\NormalTok{(model; nsim}\OperatorTok{=}\FloatTok{1000}\NormalTok{, paths}\OperatorTok{=}\ConstantTok{true}\NormalTok{, data}\OperatorTok{=}\ConstantTok{false}\NormalTok{)}
\end{Highlighting}
\end{Shaded}

The simulation uses Gillespie's algorithm for exact stochastic
simulation of the multistate process:

\begin{enumerate}
\def\labelenumi{\arabic{enumi}.}
\tightlist
\item
  Given current state \(j\) and entry time, sample sojourn time \(U\)
  from the holding time distribution
\item
  Sample destination state \(k\) from the conditional distribution given
  transition occurred
\item
  Update state to \(k\) and repeat until absorption or censoring
\end{enumerate}

\subsection{Generating Panel Data}\label{generating-panel-data}

To generate panel data (interval-censored observations):

\begin{Shaded}
\begin{Highlighting}[]
\CommentTok{\# Create observation schedule template}
\NormalTok{template }\OperatorTok{=} \FunctionTok{DataFrame}\NormalTok{(}
\NormalTok{    id }\OperatorTok{=} \FunctionTok{repeat}\NormalTok{(}\FloatTok{1}\OperatorTok{:}\FloatTok{100}\NormalTok{, inner}\OperatorTok{=}\FloatTok{5}\NormalTok{),}
\NormalTok{    tstart }\OperatorTok{=} \FunctionTok{repeat}\NormalTok{([}\FloatTok{0.0}\NormalTok{, }\FloatTok{1.0}\NormalTok{, }\FloatTok{2.0}\NormalTok{, }\FloatTok{3.0}\NormalTok{, }\FloatTok{4.0}\NormalTok{], }\FloatTok{100}\NormalTok{),}
\NormalTok{    tstop }\OperatorTok{=} \FunctionTok{repeat}\NormalTok{([}\FloatTok{1.0}\NormalTok{, }\FloatTok{2.0}\NormalTok{, }\FloatTok{3.0}\NormalTok{, }\FloatTok{4.0}\NormalTok{, }\FloatTok{5.0}\NormalTok{], }\FloatTok{100}\NormalTok{),}
\NormalTok{    statefrom }\OperatorTok{=} \FloatTok{1}\NormalTok{,}
\NormalTok{    stateto }\OperatorTok{=} \FloatTok{1}\NormalTok{,}
\NormalTok{    obstype }\OperatorTok{=} \FloatTok{2}
\NormalTok{)}

\NormalTok{model }\OperatorTok{=} \FunctionTok{multistatemodel}\NormalTok{(h12, h13, h23; data}\OperatorTok{=}\NormalTok{template)}
\FunctionTok{set\_parameters!}\NormalTok{(model, params)}

\CommentTok{\# Simulate data at observation times}
\NormalTok{sim\_data }\OperatorTok{=} \FunctionTok{simulate}\NormalTok{(model; nsim}\OperatorTok{=}\FloatTok{1}\NormalTok{, paths}\OperatorTok{=}\ConstantTok{false}\NormalTok{, data}\OperatorTok{=}\ConstantTok{true}\NormalTok{)}
\end{Highlighting}
\end{Shaded}

\section{Inference}\label{sec-inference}

\subsection{The fit Function}\label{the-fit-function}

Model fitting uses a unified interface:

\begin{Shaded}
\begin{Highlighting}[]
\NormalTok{fitted }\OperatorTok{=} \FunctionTok{fit}\NormalTok{(model;}
\NormalTok{    verbose }\OperatorTok{=} \ConstantTok{true}\NormalTok{,}
\NormalTok{    compute\_vcov }\OperatorTok{=} \ConstantTok{true}\NormalTok{,}
\NormalTok{    vcov\_type }\OperatorTok{=} \OperatorTok{:}\NormalTok{ij,}
\NormalTok{    maxiter }\OperatorTok{=} \FloatTok{100}\NormalTok{,}
\NormalTok{    tol }\OperatorTok{=} \FloatTok{1e{-}4}
\NormalTok{)}
\end{Highlighting}
\end{Shaded}

The function automatically dispatches to the appropriate algorithm based
on model and data characteristics:

\begin{itemize}
\tightlist
\item
  \textbf{Exact data}: Direct maximum likelihood via gradient-based
  optimization
\item
  \textbf{Panel data, Markov model}: Matrix exponential likelihood
\item
  \textbf{Panel data, semi-Markov model}: Monte Carlo EM algorithm
\end{itemize}

\subsection{Direct Maximum Likelihood}\label{direct-maximum-likelihood}

For exactly observed data, the log-likelihood
(Equation~\ref{eq-exact-lik}) is maximized directly using L-BFGS with
gradients computed via automatic differentiation
\citep{revels2016forwarddiff}.

When hazards are \emph{separable} (no shared parameters across
transitions), each transition-specific likelihood is optimized
independently, improving computational efficiency.

\subsection{Matrix Exponential
Likelihood}\label{matrix-exponential-likelihood}

For Markov models with panel data, the transition probability matrix is
computed via matrix exponential. \textbf{MultistateModels.jl} uses
uniformization \citep{jensen1953markoff} for numerical stability: \[
\exp(\mathbf{Q} \Delta t) = e^{-\mu \Delta t} \sum_{n=0}^{\infty} 
\frac{(\mu \Delta t)^n}{n!} \mathbf{R}^n
\] where \(\mu \geq \max_j |Q_{jj}|\) and
\(\mathbf{R} = \mathbf{I} + \mathbf{Q}/\mu\) is a valid transition
matrix.

\subsection{Monte Carlo EM for Semi-Markov Models}\label{sec-mcem}

For semi-Markov models with panel data, we implement the MCEM algorithm
of \citet{morsomme2025semimarkov}. The key insight is to treat the
unobserved state transitions as missing data and iterate:

\textbf{E-step}: Sample latent paths \(\{\pi^{(m)}\}_{m=1}^M\) from the
conditional distribution given observations and current parameters
\(\theta^{(t)}\).

\textbf{M-step}: Update parameters by maximizing the expected
complete-data log-likelihood:
\begin{equation}\phantomsection\label{eq-mstep}{
\theta^{(t+1)} = \arg\max_\theta \sum_{m=1}^M w^{(m)} \log p(\pi^{(m)} \mid \theta)
}\end{equation} where \(w^{(m)}\) are importance weights.

\subsubsection{Importance Sampling}\label{importance-sampling}

Direct sampling from the conditional path distribution is challenging.
Instead, we use importance sampling with a proposal distribution
\(q(\pi \mid \theta^{(t)})\):

\textbf{Markov proposal} (\texttt{MarkovProposal}): Sample from a Markov
surrogate model fitted to match marginal transition probabilities.
Efficient but may have high variance for strongly non-Markov processes.

\textbf{Phase-type proposal} (\texttt{PhaseTypeProposal}): Expand the
state space using phase-type distributions that better approximate the
semi-Markov dynamics. Provides lower-variance importance weights at
increased computational cost.

\subsubsection{Forward-Filtering
Backward-Sampling}\label{forward-filtering-backward-sampling}

Paths are sampled using the FFBS algorithm
\citep{cappe2005inference, alaa2018hasmm}:

\begin{enumerate}
\def\labelenumi{\arabic{enumi}.}
\tightlist
\item
  \textbf{Forward pass}: Compute filtering distributions
  \(P(X(t_m) \mid Y_{1:m})\) at each observation time using
  uniformization
\item
  \textbf{Backward pass}: Sample state trajectory in reverse time,
  conditioning on future sampled states
\end{enumerate}

This produces exact samples from the proposal distribution, with
importance weights correcting for the discrepancy from the target
semi-Markov model.

\subsubsection{Sampling Importance
Resampling}\label{sampling-importance-resampling}

To improve sample diversity and reduce weight degeneracy, we employ
Sampling Importance Resampling (SIR) with Pareto-smoothed importance
sampling (PSIS) diagnostics \citep{vehtari2024psis}:

\begin{enumerate}
\def\labelenumi{\arabic{enumi}.}
\tightlist
\item
  Generate initial pool of \(N_{\text{pool}}\) weighted paths
\item
  Compute PSIS-smoothed weights and diagnostic \(\hat{k}\)
\item
  Resample using Latin hypercube sampling for diversity
\item
  Adjust pool size adaptively based on effective sample size
\end{enumerate}

The Pareto-\(\hat{k}\) diagnostic indicates proposal quality: -
\(\hat{k} < 0.5\): Excellent proposal - \(0.5 \leq \hat{k} < 0.7\): Good
proposal\\
- \(0.7 \leq \hat{k} < 1.0\): Acceptable but monitor -
\(\hat{k} \geq 1.0\): Poor proposal, consider alternatives

\subsubsection{SQUAREM Acceleration}\label{squarem-acceleration}

We accelerate EM convergence using the SQUAREM algorithm
\citep{varadhan2008squarem}, which applies a quasi-Newton step based on
two successive EM updates. This typically reduces iteration count by
50-80\% compared to standard EM.

\begin{figure}

\centering{

\pandocbounded{\includegraphics[keepaspectratio]{figures/convergence.pdf}}

}

\caption{\label{fig-convergence}MCEM convergence diagnostics. Top: Trace
plots of parameter estimates showing stabilization. Bottom:
Log-likelihood ascent over iterations.}

\end{figure}%

\subsection{Variance Estimation}\label{sec-variance}

\subsubsection{Model-Based Variance}\label{model-based-variance}

The asymptotic variance-covariance matrix is estimated via the observed
information matrix: \[
\widehat{\text{Var}}(\hat{\theta}) = \left[-\nabla^2 \log \mathcal{L}(\hat{\theta})\right]^{-1}
\]

For MCEM, we use Louis' formula \citep{louis1982observed} to account for
the missing data: \[
\mathcal{I}_{\text{obs}}(\theta) = \mathcal{I}_{\text{comp}}(\theta) - 
\mathcal{I}_{\text{miss}}(\theta)
\]

\subsubsection{Infinitesimal Jackknife (Sandwich)
Variance}\label{infinitesimal-jackknife-sandwich-variance}

The infinitesimal jackknife provides a robust variance estimate
\citep{jaeckel1972infinitesimal}: \[
\widehat{\text{Var}}_{\text{IJ}}(\hat{\theta}) = 
\mathbf{H}^{-1} \left(\sum_{i=1}^n \mathbf{s}_i \mathbf{s}_i^\top\right) \mathbf{H}^{-1}
\] where \(\mathbf{H}\) is the Hessian and \(\mathbf{s}_i\) is the score
contribution from subject \(i\). This is equivalent to the sandwich
estimator and provides valid inference under model misspecification.

\subsubsection{Jackknife Variance}\label{jackknife-variance}

Leave-one-out jackknife variance estimation is also available: \[
\widehat{\text{Var}}_{\text{JK}}(\hat{\theta}) = \frac{n-1}{n} 
\sum_{i=1}^n (\hat{\theta}_{(-i)} - \bar{\theta})(\hat{\theta}_{(-i)} - \bar{\theta})^\top
\] where \(\hat{\theta}_{(-i)}\) is the estimate with subject \(i\)
removed.

\section{Model Selection and Regularization}\label{sec-model-selection}

\subsection{Information Criteria}\label{information-criteria}

Standard information criteria are available:

\begin{Shaded}
\begin{Highlighting}[]
\FunctionTok{AIC}\NormalTok{(fitted)}
\FunctionTok{BIC}\NormalTok{(fitted)}
\end{Highlighting}
\end{Shaded}

For MCEM, the marginal log-likelihood is estimated using importance
sampling weights from the final iteration.

\subsection{Cross-Validation (Planned)}\label{cross-validation-planned}

\emph{K-fold cross-validation for multistate models is planned for a
future release. The implementation will account for the longitudinal
structure through blocked CV schemes that keep each subject's
observations together.}

\subsection{Penalized Splines
(Planned)}\label{penalized-splines-planned}

\emph{Integration with penalized spline (P-spline) methods is planned,
enabling automatic smoothing parameter selection through restricted
maximum likelihood (REML) or cross-validation.}

\subsection{Neural ODE Hazards
(Planned)}\label{neural-ode-hazards-planned}

\emph{Future versions will support neural ordinary differential equation
(ODE) hazard specifications, allowing highly flexible hazard functions
learned from data while maintaining interpretable integration with the
multistate framework.}

\section{Computational Details}\label{sec-computational}

\subsection{Julia Implementation}\label{julia-implementation}

\textbf{MultistateModels.jl} is implemented in Julia
\citep{bezanson2017julia}, chosen for:

\begin{itemize}
\tightlist
\item
  \textbf{Performance}: JIT compilation provides C-like speed
\item
  \textbf{Expressiveness}: High-level syntax for complex algorithms
\item
  \textbf{Automatic differentiation}: Native AD via ForwardDiff.jl
  \citep{revels2016forwarddiff}
\item
  \textbf{Package ecosystem}: Integration with Optim.jl
  \citep{mogensen2018optim}, DifferentialEquations.jl
  \citep{rackauckas2017differentialequations}
\end{itemize}

\subsection{Key Dependencies}\label{key-dependencies}

\begin{itemize}
\tightlist
\item
  \texttt{ForwardDiff.jl}: Forward-mode automatic differentiation for
  gradients
\item
  \texttt{Optim.jl}: Optimization algorithms (L-BFGS, Newton)
\item
  \texttt{BSplineKit.jl}: B-spline basis functions
\item
  \texttt{DataFrames.jl}: Tabular data handling
\item
  \texttt{StatsModels.jl}: Formula interface for covariates
\end{itemize}

\subsection{Numerical Stability}\label{numerical-stability}

Several techniques ensure numerical stability:

\begin{itemize}
\tightlist
\item
  \textbf{Log-space arithmetic}: All likelihood computations use
  log-probabilities
\item
  \textbf{Uniformization}: Matrix exponentials computed via
  probabilistic interpretation
\item
  \textbf{Underflow protection}: Automatic rescaling of small
  probabilities
\end{itemize}

\subsection{Parallel Computing}\label{parallel-computing}

Subject-level likelihood contributions are computed in parallel using
Julia's threading:

\begin{Shaded}
\begin{Highlighting}[]
\NormalTok{fitted }\OperatorTok{=} \FunctionTok{fit}\NormalTok{(model; parallel}\OperatorTok{=}\ConstantTok{true}\NormalTok{)}
\end{Highlighting}
\end{Shaded}

\section{Model Outputs and Diagnostics}\label{sec-outputs}

\subsection{Extracting Results}\label{extracting-results}

\begin{Shaded}
\begin{Highlighting}[]
\CommentTok{\# Parameter estimates}
\NormalTok{params }\OperatorTok{=} \FunctionTok{get\_parameters}\NormalTok{(fitted)}
\NormalTok{params\_flat }\OperatorTok{=} \FunctionTok{get\_parameters\_flat}\NormalTok{(fitted)}

\CommentTok{\# Variance{-}covariance matrix}
\NormalTok{vcov }\OperatorTok{=} \FunctionTok{get\_vcov}\NormalTok{(fitted)}

\CommentTok{\# Confidence intervals}
\FunctionTok{confint}\NormalTok{(fitted, level}\OperatorTok{=}\FloatTok{0.95}\NormalTok{)}
\end{Highlighting}
\end{Shaded}

\subsection{Convergence Diagnostics}\label{convergence-diagnostics}

For MCEM fits, convergence diagnostics are available:

\begin{Shaded}
\begin{Highlighting}[]
\CommentTok{\# Access convergence records}
\NormalTok{records }\OperatorTok{=}\NormalTok{ fitted.ConvergenceRecords}

\CommentTok{\# Pareto{-}k diagnostics}
\NormalTok{records.psis\_pareto\_k}

\CommentTok{\# ESS trajectory}
\NormalTok{records.ess\_history}

\CommentTok{\# Parameter trace}
\NormalTok{records.parameter\_history}
\end{Highlighting}
\end{Shaded}

Good convergence is indicated by: - Stabilized parameter estimates -
Pareto-\(\hat{k} < 0.7\) for most iterations - ESS close to target

\section{Examples}\label{sec-examples}

\subsection{Illness-Death Model with Panel
Data}\label{illness-death-model-with-panel-data}

We demonstrate fitting a semi-Markov illness-death model with Weibull
hazards to panel data.

\begin{Shaded}
\begin{Highlighting}[]
\NormalTok{\#| eval: false}
\NormalTok{using MultistateModels}
\NormalTok{using DataFrames}
\NormalTok{using Random}

\NormalTok{Random.seed!(12345)}

\NormalTok{\# Define hazards}
\NormalTok{h12 = Hazard(:wei, 1, 2)  \# Healthy → Ill}
\NormalTok{h23 = Hazard(:wei, 2, 3)  \# Ill → Dead  }
\NormalTok{h13 = Hazard(:wei, 1, 3)  \# Healthy → Dead}

\NormalTok{\# Generate panel data template}
\NormalTok{n\_subjects = 200}
\NormalTok{obs\_times = [0.0, 2.0, 4.0, 6.0, 8.0, 10.0]}
\NormalTok{n\_obs = length(obs\_times) {-} 1}

\NormalTok{panel\_template = DataFrame(}
\NormalTok{    id = repeat(1:n\_subjects, inner=n\_obs),}
\NormalTok{    tstart = repeat(obs\_times[1:end{-}1], n\_subjects),}
\NormalTok{    tstop = repeat(obs\_times[2:end], n\_subjects),}
\NormalTok{    statefrom = 1,}
\NormalTok{    stateto = 1,}
\NormalTok{    obstype = 2}
\NormalTok{)}

\NormalTok{\# Create model for simulation}
\NormalTok{sim\_model = multistatemodel(h12, h23, h13; }
\NormalTok{                            data=panel\_template, }
\NormalTok{                            surrogate=:markov)}

\NormalTok{\# Set true parameters (log{-}shape, log{-}scale for Weibull)}
\NormalTok{set\_parameters!(sim\_model, (}
\NormalTok{    h12 = [log(1.5), log(5.0)],   \# Shape=1.5, Scale=5.0}
\NormalTok{    h23 = [log(1.8), log(3.0)],   \# Shape=1.8, Scale=3.0}
\NormalTok{    h13 = [log(1.2), log(10.0)]   \# Shape=1.2, Scale=10.0}
\NormalTok{))}

\NormalTok{\# Simulate data}
\NormalTok{sim\_result = simulate(sim\_model; nsim=1, paths=false, data=true)}
\NormalTok{panel\_data = sim\_result[1, 1]}

\NormalTok{\# Fit model using MCEM}
\NormalTok{fit\_model = multistatemodel(h12, h23, h13;}
\NormalTok{                            data=panel\_data,}
\NormalTok{                            surrogate=:markov)}

\NormalTok{fitted = fit(fit\_model;}
\NormalTok{    verbose = true,}
\NormalTok{    compute\_vcov = true,}
\NormalTok{    vcov\_type = :ij,}
\NormalTok{    maxiter = 50,}
\NormalTok{    tol = 0.01,}
\NormalTok{    ess\_target\_initial = 50}
\NormalTok{)}

\NormalTok{\# Results}
\NormalTok{println("Fitted parameters:")}
\NormalTok{println(get\_parameters(fitted))}
\NormalTok{println("\textbackslash{}nStandard errors:")}
\NormalTok{println(sqrt.(diag(get\_vcov(fitted))))}
\end{Highlighting}
\end{Shaded}

\subsection{Model with Covariates}\label{model-with-covariates}

\begin{Shaded}
\begin{Highlighting}[]
\NormalTok{\#| eval: false}
\NormalTok{\# Hazard with treatment effect}
\NormalTok{h12\_cov = Hazard(@formula(0 \textasciitilde{} treatment), :wei, 1, 2)}
\NormalTok{h23\_cov = Hazard(@formula(0 \textasciitilde{} treatment), :wei, 2, 3)}
\NormalTok{h13\_cov = Hazard(@formula(0 \textasciitilde{} treatment), :wei, 1, 3)}

\NormalTok{\# Data must include treatment column}
\NormalTok{panel\_data.treatment = repeat([0, 1], inner=n\_subjects÷2 * n\_obs)}

\NormalTok{model\_cov = multistatemodel(h12\_cov, h23\_cov, h13\_cov;}
\NormalTok{                            data=panel\_data,}
\NormalTok{                            surrogate=:markov)}

\NormalTok{fitted\_cov = fit(model\_cov; verbose=true, compute\_vcov=true)}

\NormalTok{\# Treatment hazard ratios}
\NormalTok{params = get\_parameters(fitted\_cov)}
\NormalTok{hr\_12 = exp(params.h12[end])  \# Last parameter is treatment effect}
\NormalTok{println("Treatment HR for 1→2: ", round(hr\_12, digits=3))}
\end{Highlighting}
\end{Shaded}

\section{Comparison with Existing Software}\label{sec-comparison}

\subsection{Feature Comparison}\label{feature-comparison}

Table~\ref{tbl-detailed-comparison} provides a detailed feature
comparison.

\begin{longtable}[]{@{}
  >{\raggedright\arraybackslash}p{(\linewidth - 16\tabcolsep) * \real{0.1000}}
  >{\centering\arraybackslash}p{(\linewidth - 16\tabcolsep) * \real{0.2333}}
  >{\centering\arraybackslash}p{(\linewidth - 16\tabcolsep) * \real{0.1111}}
  >{\centering\arraybackslash}p{(\linewidth - 16\tabcolsep) * \real{0.0556}}
  >{\centering\arraybackslash}p{(\linewidth - 16\tabcolsep) * \real{0.0889}}
  >{\centering\arraybackslash}p{(\linewidth - 16\tabcolsep) * \real{0.0556}}
  >{\centering\arraybackslash}p{(\linewidth - 16\tabcolsep) * \real{0.1333}}
  >{\centering\arraybackslash}p{(\linewidth - 16\tabcolsep) * \real{0.1111}}
  >{\centering\arraybackslash}p{(\linewidth - 16\tabcolsep) * \real{0.1111}}@{}}
\caption{Detailed feature comparison. --- indicates not applicable for
non-parametric methods.}\label{tbl-detailed-comparison}\tabularnewline
\toprule\noalign{}
\begin{minipage}[b]{\linewidth}\raggedright
Feature
\end{minipage} & \begin{minipage}[b]{\linewidth}\centering
MultistateModels.jl
\end{minipage} & \begin{minipage}[b]{\linewidth}\centering
survival
\end{minipage} & \begin{minipage}[b]{\linewidth}\centering
msm
\end{minipage} & \begin{minipage}[b]{\linewidth}\centering
mstate
\end{minipage} & \begin{minipage}[b]{\linewidth}\centering
etm
\end{minipage} & \begin{minipage}[b]{\linewidth}\centering
SemiMarkov
\end{minipage} & \begin{minipage}[b]{\linewidth}\centering
flexsurv
\end{minipage} & \begin{minipage}[b]{\linewidth}\centering
icmstate
\end{minipage} \\
\midrule\noalign{}
\endfirsthead
\toprule\noalign{}
\begin{minipage}[b]{\linewidth}\raggedright
Feature
\end{minipage} & \begin{minipage}[b]{\linewidth}\centering
MultistateModels.jl
\end{minipage} & \begin{minipage}[b]{\linewidth}\centering
survival
\end{minipage} & \begin{minipage}[b]{\linewidth}\centering
msm
\end{minipage} & \begin{minipage}[b]{\linewidth}\centering
mstate
\end{minipage} & \begin{minipage}[b]{\linewidth}\centering
etm
\end{minipage} & \begin{minipage}[b]{\linewidth}\centering
SemiMarkov
\end{minipage} & \begin{minipage}[b]{\linewidth}\centering
flexsurv
\end{minipage} & \begin{minipage}[b]{\linewidth}\centering
icmstate
\end{minipage} \\
\midrule\noalign{}
\endhead
\bottomrule\noalign{}
\endlastfoot
\textbf{Data Types} & & & & & & & & \\
Exact observations & ✓ & ✓ & ✓ & ✓ & ✓ & ✓ & ✓ & ✗ \\
Panel/interval-censored & ✓ & ✗ & ✓ & ✗ & ✗ & ✗ & ✗ & ✓ \\
Right censoring & ✓ & ✓ & ✓ & ✓ & ✓ & ✓ & ✓ & ✓ \\
\textbf{Model Types} & & & & & & & & \\
Markov multistate & ✓ & ✗† & ✓ & ✓ & ✓ & ✗ & ✓ & ✓ \\
Semi-Markov multistate & ✓ & ✗† & ✗ & ✗ & ✗ & ✓ & ✓ & ✗ \\
Hidden Markov & ✗ & ✗ & ✓ & ✗ & ✗ & ✗ & ✗ & ✗ \\
\textbf{Inference Approach} & & & & & & & & \\
Non-parametric & ✗ & ✓ & ✗ & ✓ & ✓ & ✗ & ✗ & ✓ \\
Parametric & ✓ & ✗ & ✓ & ✗ & ✗ & ✓ & ✓ & ✗ \\
Semi-parametric (splines/Cox) & ✓ & ✓ & ✗ & ✓ & ✗ & ✗ & ✓ & ✗ \\
\textbf{Hazard Families} & & & & & & & & \\
Exponential & ✓ & ✓ & --- & --- & ✓ & ✓ & --- & \\
Weibull & ✓ & ✓ & --- & --- & ✓ & ✓ & --- & \\
Gompertz & ✓ & ✓ & --- & --- & ✗ & ✓ & --- & \\
Splines & ✓ & ✗ & --- & --- & ✗ & ✓ & --- & \\
Phase-type & ✓ & ✗ & --- & --- & ✗ & ✗ & --- & \\
\textbf{Covariate Effects} & & & & & & & & \\
Proportional hazards & ✓ & ✓ & ✓ & ✗ & ✓ & ✓ & ✗ & \\
Time-varying covariates & ✓ & ✓ & ✓ & ✗ & ✓ & ✓ & ✗ & \\
\textbf{Variance Estimation} & & & & & & & & \\
Model-based & ✓ & ✓ & ✓ & ✓ & ✓ & ✓ & ✓ & \\
Sandwich/IJ & ✓ & ✗ & ✗ & ✗ & ✗ & ✗ & ✗ & \\
Jackknife & ✓ & ✓ & ✗ & ✗ & ✗ & ✗ & ✗ & \\
\end{longtable}

\subsection{Methodological
Distinctions}\label{methodological-distinctions}

\textbf{vs.~msm}: While \textbf{msm} is the standard tool for panel data
multistate models, it is fundamentally limited to the Markov assumption.
Our package extends the panel data capability to semi-Markov models
through MCEM. For Markov models with panel data, both packages should
give equivalent results.

\textbf{vs.~survival}: The \textbf{survival} package is the standard for
semi-parametric (Cox regression) and non-parametric (Kaplan-Meier)
survival analysis. While it handles competing risks, it is not designed
for general multistate models and cannot accommodate
panel/interval-censored data.

\textbf{vs.~mstate and etm}: These packages provide non-parametric
estimation (Aalen-Johansen estimator, Nelson-Aalen cumulative hazard)
and Cox regression-based inference for multistate models with exact
transition times. They excel when model-free estimation is desired and
sample sizes are sufficient. However, they cannot be used with
panel/interval-censored data and do not provide the efficiency gains of
fully parametric methods. \textbf{MultistateModels.jl} focuses on
parametric and semi-parametric (B-spline) inference, complementing
rather than replacing these tools.

\textbf{vs.~SemiMarkov}: The \textbf{SemiMarkov} package fits
semi-Markov models but \textbf{requires exact observation of transition
times}. This is a fundamental limitation---it cannot be used when only
the state at scheduled observation times is known, which is the common
situation in clinical studies.

\textbf{vs.~flexsurv}: The \textbf{flexsurv} package offers excellent
flexibility for parametric survival models including semi-Markov
multistate models with spline-based hazards and clock-reset (sojourn
time) formulations. However, it \textbf{requires exact transition times}
and cannot handle panel/interval-censored observations where the timing
of state changes is unknown.

\subsection{When to Use Each Package}\label{when-to-use-each-package}

\begin{itemize}
\tightlist
\item
  \textbf{Use survival} for Cox regression or Kaplan-Meier estimation
  with single-endpoint or competing risks data
\item
  \textbf{Use mstate or etm} for non-parametric or Cox regression-based
  multistate models with exact transition times, especially for
  exploratory analysis or when avoiding parametric distributional
  assumptions is critical
\item
  \textbf{Use msm} for Markov models with panel data when the
  exponential sojourn assumption is reasonable, or for hidden Markov
  models
\item
  \textbf{Use SemiMarkov} for semi-Markov models with exact times when
  Weibull or exponentiated Weibull distributions are appropriate
\item
  \textbf{Use flexsurv} for semi-Markov or Markov models with exact
  transition times, especially when flexible spline-based hazards are
  desired
\item
  \textbf{Use icmstate} for non-parametric Markov models with
  panel/interval-censored data
\item
  \textbf{Use MultistateModels.jl} for parametric/semi-parametric
  semi-Markov models with panel data, phase-type distributions, or when
  robust variance estimation (sandwich, jackknife) is needed
\end{itemize}

\section{Summary and Future Directions}\label{sec-conclusion}

\subsection{Summary}\label{summary}

\textbf{MultistateModels.jl} provides a comprehensive framework for
fitting parametric and semi-parametric multistate models in Julia, with
particular strength in handling semi-Markov models with panel data---a
capability not available in existing software. The package implements
the MCEM methodology of \citet{morsomme2025semimarkov} with efficient
importance sampling, SQUAREM acceleration, and multiple variance
estimation approaches.

\subsection{Limitations}\label{limitations}

Current limitations include:

\begin{itemize}
\tightlist
\item
  No hidden state support (all states assumed observable at panel times)
\item
  Single initial state assumed (though easily modified)
\item
  No Bayesian inference (frequentist MLE only)
\end{itemize}

\subsection{Future Directions}\label{future-directions}

Planned extensions include:

\begin{enumerate}
\def\labelenumi{\arabic{enumi}.}
\tightlist
\item
  \textbf{Penalized splines} with automatic smoothing selection
\item
  \textbf{Cross-validation} framework for model selection
\item
  \textbf{Neural ODE hazards} for highly flexible specifications
\item
  \textbf{Bayesian inference} via Hamiltonian Monte Carlo
\item
  \textbf{GPU acceleration} for large-scale applications
\end{enumerate}

\subsection{Availability}\label{availability}

\textbf{MultistateModels.jl} is available at
\url{https://github.com/fintzij/MultistateModels.jl} under the MIT
license.

\section{Acknowledgments}\label{acknowledgments}

We thank Tom Schiemsky and Andrew Beck for contributions to the
methodology and testing.

\section*{References}\label{references}
\addcontentsline{toc}{section}{References}

\renewcommand{\bibsection}{}
\bibliography{references.bib}

\section{Appendix A: Uniformization
Algorithm}\label{sec-appendix-uniformization}

The uniformization algorithm computes the matrix exponential
\(\mathbf{P}(\Delta t) = \exp(\mathbf{Q} \Delta t)\) via:

\begin{enumerate}
\def\labelenumi{\arabic{enumi}.}
\tightlist
\item
  Choose uniformization rate \(\mu \geq \max_j |Q_{jj}|\)
\item
  Construct transition matrix
  \(\mathbf{R} = \mathbf{I} + \mathbf{Q}/\mu\)
\item
  Compute truncated sum: \[
  \mathbf{P}(\Delta t) \approx e^{-\mu \Delta t} \sum_{n=0}^{N} 
  \frac{(\mu \Delta t)^n}{n!} \mathbf{R}^n
  \] where \(N\) is chosen so the Poisson tail probability is below
  tolerance.
\end{enumerate}

This representation interprets the continuous-time chain as a
discrete-time chain \(\mathbf{R}\) with Poisson-distributed number of
jumps.

\section{Appendix B: FFBS Algorithm}\label{sec-appendix-ffbs}

\textbf{Forward Filtering:}

For each observation time \(t_m\), compute: \[
\alpha_m(j) = P(X(t_m) = j \mid Y_{1:m})
\] recursively using: \[
\alpha_m(j) \propto \sum_{i} \alpha_{m-1}(i) P_{ij}(t_{m-1}, t_m) e_j(Y_m)
\] where \(e_j(Y_m)\) is the emission probability (1 if \(Y_m = j\), 0
otherwise for exact observations).

\textbf{Backward Sampling:}

Sample \(X(t_M) \sim \alpha_M\), then for \(m = M-1, \ldots, 1\): \[
X(t_m) \sim P(X(t_m) = j \mid X(t_{m+1}), Y_{1:m}) \propto 
\alpha_m(j) P_{j, X(t_{m+1})}(t_m, t_{m+1})
\]

Within-interval paths are sampled using endpoint-conditioned
uniformization \citep{hobolth2009simulation}.

\section{Appendix C: Phase-Type
Distributions}\label{sec-appendix-phasetype}

A Coxian distribution with \(n\) phases has intensity matrix: \[
\mathbf{S} = \begin{pmatrix}
-\lambda_1 & p_1 \lambda_1 & 0 & \cdots & 0 \\
0 & -\lambda_2 & p_2 \lambda_2 & \cdots & 0 \\
\vdots & & \ddots & & \vdots \\
0 & 0 & \cdots & -\lambda_{n-1} & p_{n-1}\lambda_{n-1} \\
0 & 0 & \cdots & 0 & -\lambda_n
\end{pmatrix}
\]

with absorption rates
\(\mathbf{s} = ((1-p_1)\lambda_1, \ldots, (1-p_{n-1})\lambda_{n-1}, \lambda_n)^\top\).

The density, survival, and hazard functions are: \[
f(t) = \boldsymbol{\pi} e^{\mathbf{S}t} \mathbf{s}, \quad
S(t) = \boldsymbol{\pi} e^{\mathbf{S}t} \mathbf{1}, \quad
h(t) = \frac{f(t)}{S(t)}
\] where \(\boldsymbol{\pi} = (1, 0, \ldots, 0)\) for Coxian
distributions starting in phase 1.

\section{Appendix D: Variance Estimation
Details}\label{sec-appendix-variance}

\subsection{Louis' Formula}\label{louis-formula}

For the observed information in EM: \[
\mathcal{I}_{\text{obs}}(\theta) = -E_{\theta}[\nabla^2 \log p(Y, Z \mid \theta) \mid Y] 
- \text{Var}_{\theta}[\nabla \log p(Y, Z \mid \theta) \mid Y]
\]

The first term is the expected complete-data information; the second
measures information loss due to missing data.

\subsection{Infinitesimal Jackknife}\label{infinitesimal-jackknife}

For subject \(i\), the influence function is: \[
\mathbf{IC}_i = \mathbf{H}^{-1} \mathbf{s}_i
\] where
\(\mathbf{s}_i = \nabla_\theta \log \mathcal{L}_i(\hat{\theta})\).

The IJ variance is: \[
\widehat{\text{Var}}_{\text{IJ}} = \sum_{i=1}^n \mathbf{IC}_i \mathbf{IC}_i^\top
= \mathbf{H}^{-1} \left(\sum_{i=1}^n \mathbf{s}_i \mathbf{s}_i^\top\right) \mathbf{H}^{-1}
\]

This equals the sandwich estimator and is consistent even under model
misspecification.





\end{document}
